\documentclass[11pt, a4paper]{paper}
\usepackage{amsmath}
\usepackage{graphicx}
\usepackage{subfig}

\title {\bf Jaal:: Engineering A High Quality All Quads Surface Mesher } 
\author {
Chaman Singh Verma\thanks{csverma@cs.wisc.edu}, Tim Taugtes \thanks{tautges@mcs.anl.gov} }

\begin{document}
\maketitle
\abstract { 
Applications of All-Quad surface mesh are abound, but to the best of our knowledge, there does not
exist any reliable public domain All-Quads meshing implementation even for complex 2D domains. Among 
various techniques that have been proposed in the past, probably {\em triangle to quad} transformation is
the most intuitive, appealing, and most importantly, robust.

The main objective of the work is to relax convexity condition from Suneeta's tree matching algorithm
to produce an extremely fast and robust all -quad meshing algorithm. After generating topologically
an {\em All Quad Mesh} using matching algorithm, we improve the resulting mesh using simple clean-up operations,
and quadrilateral shape optimization libaries.

This All Quad Mesh algorithm is part of MOAB toolkit, which is an open source library supporting mesh infrastructure for 
petacscale field simulations.

We demonstrate the effectivness of our algorithm for both CAD and non-CAD geometrical models.
Initial experiments shows that our implementation is at least 100 times faster than the original
implementation of Suneeta's algorithm for generating topological correct mesh. Performance with
all the cleanup operations and optimization is also within some reasonable limits for some of the 
benchmarking models.
}

\section { Introduction }
Quadrilateral mesh generation have been used since late 70s for many commerical CFD codes. The importance
of unstructured quadrilateral mesh for the complex geometries is well known, but it is extremely surprsing 
know( to the best of our knowledge) that there does not exist any open source All-Quad surface meshing algorithm at par with high visible unstructured triangular mesh {\em Triangle }.

Quadrilateral mesh generation do not have the theoretical strong backup as the trinagle mesh generation, 
still we strongly believe that it is possible to engineering an All-Quad meshing algorithm with some 
set heurtistics, known algorithms and optimization libraries that can produce an extremely high quality
quadrilateral mesh for most of models of practical importance, if not all. 

In this work, we focus on {\em Triangle to Quad} transformation approach because of its simplicity and
robustness. Qmorph implementation has similar final goals, it generates high quality quadrilateral mesh
with sweeping {\em Advancing front } to transform and cleanup at the same step. This approach suuffers
from one of the serious problem quality mesh in the regions of front collisions. Therefore, we strongly
believe in the second approach i.e. seperate generating topological correct quadrilateral mesh from the
quality improvement. For this reason, we follow our implementations on the idea of using graph matching
for quadrilateral mesh generation, first proposed by Suneeta et. al. In fact, our work is so much influenced
by her work, that for most of the theoretical proofs and implementation ideas are taken from her work.  

The rest of the paper is organized as follows: Section 1, describes Triangle to Quad tranformation using graph and tree matching algorithms. Section 2, descibes improving topologically correct but geometrically unacceptable quadrilateral mesh. Section 3 describes results for both CAD and non-CAD geometrical models.
Section 4. describes implementation details and finally in Section 5, we conclude this work.
 \begin{figure}
 \begin{center}
 \includegraphics[scale=0.4]{quadflow.eps}
 \caption{There are seven independent components in developing high quality all-quad meshing algorithm.}
 \end{center}
 \label{fig:flowchart}
\end{figure} 
\section {Previous work on all-quad meshing algorithms}

% Advancing front Recombination ( QMorph ) 
% ... ( Paving )
% Circle packing 
%Spectral Patches
% Global Parameterization
%Disk patches
% Graph Matching.
% Statement of Observations
%   Complexity and Functionality.
% Graph Matching 
%  Edmonds  
%  Suneeta
% Jaal: State its complexity.
%     Relaxation of Convexity
%     Bounded Quality.
% Implementation
%      Comp. Inden. of ( Tri, Global and 3D Alg.).
% Results
%      Fastest Known Implementation.
%      Robustness 2D/3D
% Conclusions
% Future Work.
%      Suneeta's Complexity estimation.


`
















\section { Graph Matching Algorithm }
\begin{figure}
 \begin{center}
 \includegraphics[scale=0.5]{graphtree.eps}
 %\includegraphics[scale=0.4]{dfs.eps}
 %\includegraphics[scale=0.4]{bfs.eps}
 \caption{Converting Graph into a tree.}
 \end{center}
 \label{fig:graph_tree}
\end{figure} 
\section { Topological Clean-up}
\subsection{ Quadrilateral Convexity }
\subsection{ Basic Operations}
\subsubsection{ Face Closing }
\subsubsection{ Face Opening }
\subsubsection{ Face Swapping }
\subsection { Advancing front clean-up}
\subsection { Boundary improvement}
\subsection { Volume Preserving Laplacian Smoothing}
\section {Results}
\section { Implementation details }

\section {Conclusions}
\end{document}
